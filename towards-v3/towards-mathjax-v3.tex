\documentclass[]{amsart}
\usepackage{tabularx}
\usepackage{amssymb,amsmath}
  \usepackage[unicode=true]{hyperref}
\hypersetup{breaklinks=true,
            pdfauthor={Peter Krautzberger; Davide Cervone; Volker Sorge},
            pdftitle={Towards MathJax v3.0},
            colorlinks=true,
            citecolor=blue,
            urlcolor=blue,
            linkcolor=magenta,
            pdfborder={0 0 0}}
\urlstyle{same}  % don't use monospace font for urls
\setlength{\parindent}{0pt}
\setlength{\parskip}{6pt plus 2pt minus 1pt}
\setlength{\emergencystretch}{3em}  % prevent overfull lines
\setcounter{secnumdepth}{0}

\title{Towards MathJax v3.0}
\author{Peter Krautzberger \and Davide Cervone \and Volker Sorge}
\date{}

\begin{document}
\maketitle

\section{Executive Summary}\label{executive-summary}

Since the release of v1.0 in 2010, MathJax has become the de-facto
standard for rendering mathematics on the web. While MathJax's various
input and output components have evolved over the years, MathJax's core
component remained fixed. Today, both MathJax and browser technology
have reached a point where an extensive redesign of MathJax's core
component will enable significant improvements and ensure the long-term
viability of MathJax.

The core work of the redesign will revolve around modularity. MathJax's
original design needed to provide its own, specialized framework to
tackle the challenges specific to mathematical rendering across all
browsers (in 2010). Since then, the modularization of web technology has
made much progress while MathJax's newer components require less
complexity from its core component. The timing is right for improving
MathJax's module structure and APIs. This change should ensure MathJax
will better serve today's highly complex web development ecosystem. A
positive side effect should also be performance improvements and the
ability for developers to re-use individual components efficiently.

The redesign must be governed by MathJax's mission to provide the best
tools for the mathematical and scientific community on the web. A key
consideration is one of MathJax's original goals: to spur native MathML
implementations in browsers. Thanks to MathJax, millions of users
benefit from the advantages of MathML on the web every day, with the
MathJax CDN alone serving over 4.5 million daily visitors.
Unfortunately, browser vendors continue to lack interest in implementing
MathML while volunteer efforts have proven unreliable and of limited
scope.

Depending on whether or not we keep this original goal, we considered
two directions to guide the redesign. If native browser support for
MathML remains a core goal, we believe MathJax should focus on becoming
a modern polyfill, i.e., enable MathJax to leverage partial MathML
implementations in browsers and become as invisible as possible to
developers. If browser support for MathML ceases to be a core goal, we
believe MathJax should focus on perfecting its transformation of MathML
into HTML/CSS (and into SVG), i.e., MathJax should focus on enabling a
rendering that is fully equivalent to MathML and can be generated on
both server and client.

After careful consideration and extensive feedback from the MathJax
sponsors, the MathJax Steering Committee, as well as various experts in
the field, the MathJax Consortium follows the recommendation of its
development team to pursue the second direction as the guiding principle
for the planned revision.

To ensure that we achieve our goals, MathJax is forming a Technical
Committee of developers. This committee is being recruited from the
MathJax sponsors as well as experts from the community. In addition, the
redesign will require significant resources, including an additional
core developer.

We are grateful for the unanimous support from our MathJax sponsors to
support this effort and ensure MathJax will continue to provide the best
tools for math and science on the web.

\section{Introduction}\label{introduction}

This August marked the fifth anniversary of the release of MathJax v1.0.
In the past five years, MathJax has become the standard solution for
publishing mathematics on the web, growing from a one-developer project
to a mature project with a dedicated team, multiple contributors, and a
rich ecosystem built around it. Today, the MathJax CDN serves 4.5
million unique visitors each day and is used on thousands of websites,
including over 400 of the top 1 million websites according to
\href{http://libscore.com/\#MathJax}{libscore.com}.

We find ourselves in a very different World Wide Web today. With its
success, the expectations towards MathJax have grown. When MathJax
started, it was considered a temporary solution, to bridge the time
until browsers implemented MathML alongside HTML5. Today, that moment
seems further away than it was 5 years ago with the two leading browsers
(Internet Explorer and Chrome or
\href{https://en.wikipedia.org/wiki/Usage_share_of_web_browsers\#Summary_tables}{55-75\%
of the market}) having no plans to support MathML, even actively
removing support for MathML or for plugins that could compensate.

Moreover, web development has changed drastically over the past five
years, both in terms of tools (libraries, frameworks, platforms) and
standards (HTML5, Web APIs, ECMAScript 2015 etc), with the former often
influencing the latter. This has changed the requirements for MathJax
within modern web developer workflows and also the expectations towards
MathJax. In short, MathJax is showing its age because development had to
focus on maintenance and conservatively extending functionality.

This paper discusses the risks and benefits of overhauling a significant
portion of MathJax to enable another 5 years of successful development
and delivery.

\section{Background}\label{background}

Making extensive changes to any piece of software carries risks that
need to be outweighed by the opportunities provided. This is especially
true when considering changes that are not fully backwards compatible.
As we are considering significant (breaking) changes to MathJax's core
component, we also face a question of timing. MathJax development began
at a point where web technology improvements were ideal for re-designing
its predecessor jsMath. Given the responsibility towards MathJax's
community and MathJax's donors, we need to ensure we have reached a
similarly opportune moment in terms of existing and nascent technologies
as well as in terms of our approach towards leveraging them for long
term sustainability of the project.

\subsection{MathJax}\label{mathjax}

5 years on the web is an eternity. For an unfair comparison, MathJax is
roughly the same age as jQuery, NodeJS, or AngularJS, all of which have
passed through several iterations since. Naturally, these projects have
a much larger contributor base and financial backing. But the extensive
changes in their design indicate how drastically the web development
landscape has changed.

Still, MathJax has seen major improvements over the past 5 years. Thanks
to its modular system, many components have seen a partial or complete
redesign. Additionally, the advantage of modularity is that instead of
redesigning a monolithic piece, we could develop new components (inputs,
outputs, and their extensions) which provided the same effect as
re-writing an existing component while simultaneously keeping older
solutions available (and often improved). However, there are some
limitations within MathJax that cannot be resolved this way.

At the core of the overhaul we are proposing lies MathJax's core
component which has not been revised since v1.0.

Until very recently, a major overhaul of MathJax's internals would have
had extremely limited scope. That is, much of the core component was
simply necessary for MathJax to provide high-quality layout. However,
the progress we made in 2014/15 with the new CommonHTML output alongside
improvements in the browser landscape provide us with an opportunity to
redesign several aspects of MathJax's core component and its features.
Simply put, the internals have begun to hold MathJax development back
and we believe their disadvantages are now outweighing their advantages.

Such an overhaul will include many changes that will break current
behavior such as changing APIs or removing outdated components. A good
example is MathJax's original rendering component, the HTML-CSS output.
This original output component was designed to work reliably on IE6+ and
all other browsers and platforms that were current in 2009. This forces
MathJax to workaround significant limitations, including erratic layout
behavior, unreliable webfont integration, and low level of JavaScript
features. Not surprisingly, much of MathJax's infrastructure was
required for the HTML-CSS output given these restraints on old (now
ancient) browsers; changing the core component will result in the
removal of the HTML-CSS output while the new CommonHTML output remains
as new default. Similarly, changes to the internals will change
MathJax's APIs extensively. Of course, it is important that any breaking
changes (in particular, feature deprecation) are compensated by
improvements.

Equally important to us is that MathJax is not just another software
development project. MathJax is not developed for a particular company
or organization but instead we are driven by a mission shared by our
managing partners as well as our sponsors: to provide the best tools for
the mathematical and scientific community on the web. At the core of
MathJax has always been MathML, the web standard for math and science.
One goal of MathJax was to break the vicious cycle of ``no browser
support $\Rightarrow$ no MathML on the web $\Rightarrow$ no need for browser support $\Rightarrow$
\ldots{}''. Despite MathJax's success, we do not seem to be any closer
to native MathML support today than we were 5 years ago. In fact, we
seem further away than ever.

\subsection{Browsers in 2015}\label{browsers-in-2015}

The situation of native implementations can only be described as
confusing.

\begin{itemize}
\item
  IE/Edge lists MathML as
  \href{http://dev.modern.ie/platform/status/?filter=f3f0000bf\&search=math}{``not
  currently planned''} on its roadmap while removing support for plugins
  such as MathPlayer that could compensate.
\item
  Firefox/Gecko's MathML implementation has been the work of volunteers
  over many years and Mozilla engineers are supportive of these
  volunteer efforts. In 2014, a crowd-funded volunteer effort made some
  progress but unfortunately lasted only 6 months; no substantial
  progress has been made since.

Gecko MathML support is incomplete but in terms of plain feature
coverage not too far from MathJax (and ahead in terms of bidirectional
layout). However, layout quality is sometimes sub-par or unreliable
while common Web APIs are often not supported. It can be used in
production but often requires knowledge about its specific
implementation quirks.

\item
  Safari/WebKit has seen three attempts by three successive volunteers
  to move its implementation forward. These volunteers worked alone with
  little support from WebKit companies; all of them eventually ran out
  of time and/or money. For the past year, there has been no active work
  on WebKit's MathML support. Apple
  \href{https://developer.apple.com/safari/features/}{advertises} MathML
  support in Safari while in practice it is not usable in professional
  production. Notably,
  \href{https://trac.webkit.org/wiki/MathML}{WebKit's MathML pages} have
  not been updated in several years and
  \href{https://bugs.webkit.org/buglist.cgi?bug_status=UNCONFIRMED\&bug_status=NEW\&bug_status=ASSIGNED\&bug_status=REOPENED\&field0-0-0=product\&field0-0-1=component\&field0-0-2=alias\&field0-0-3=short_desc\&field0-0-4=status_whiteboard\&field0-0-5=content\&query_format=advanced\&type0-0-0=substring\&type0-0-1=substring\&type0-0-2=substring\&type0-0-3=substring\&type0-0-4=substring\&type0-0-5=matches\&value0-0-0=mathml\&value0-0-1=mathml\&value0-0-2=mathml\&value0-0-3=mathml\&value0-0-4=mathml\&value0-0-5=\%22mathml\%22\&order=changeddate\%20DESC\%2Cbug_status\%2Cpriority\%2Cassigned_to\%2Cbug_id\&query_based_on=}{at
  time of writing there were 136 open bugs filed under MathML}; MathML
  is also not included in the recently added
  \href{https://www.webkit.org/status.html}{``WebKit Web Platform
  Status'' dashboard}. Apple has implemented partial support in its
  proprietary VoiceOver technology.
\item
  Chrome/Blink lists MathML as
  \href{https://www.chromestatus.com/features/5240822173794304}{``no
  longer pursuing''}. Blink removed the code base for MathML that it
  inherited when forking from WebKit. The Chrome team has been
  consistent that support in Blink is not planned. While negative, it
  has been the most transparent and consistent position. In extension,
  this position applies to Opera, Vivaldi and other Blink/Chromium-based
  browsers.
\end{itemize}

It is worthwhile to note that Chrome and IE/Edge make up
\href{https://en.wikipedia.org/wiki/Usage_share_of_web_browsers\#Summary_tables}{55-75\%
of today's browser market, depending on the metric}).

Ultimately, we believe actions speak louder than words: \textbf{no
browser vendor has worked or is planning to work on implementing
MathML}.

In addition, unfunded volunteer-driven efforts have repeatedly failed to
reach the 80/20 point of the implementations. In 2013, MathJax
extensively investigated ways to reliably fund long-term, third-party
MathML browser development, however the lack of interest from browser
vendors made it difficult to pitch this idea to potential funders.

On the other hand, other components of HTML5 have been either
implemented, revised, or marked as abandoned. In addition, browser
layout engines have matured a great deal and layout has become very
reliable across browsers. Where MathJax originally could not even rely
on basic layout such as text widths being handled correctly, today's
browsers are reliable even to the point of CSS 3 implementations. This
enables new approaches to math layout. With nascent technologies, a
different path towards native math layout seems feasible.

\subsection{Web standards}\label{web-standards}

In the past year, MathJax has begun to become more pro-active regarding
web standards development to better fulfill its mission of moving math
and science notation forward on the web. By supporting Peter
Krautzberger as an invited experted to the W3C's Digital Publishing
Interest Group (DPUB IG) and the W3C's Math Working Group (Math WG),
MathJax is developing new expertise on use cases, technologies, as well
as evolving standards to align with, while in turn providing feedback to
web standards development. In the DPUB IG, Peter Krautzberger leads the
STEM task force. Overall, there is a clear frustration among STEM web
experts regarding the lack of progress for MathML. At the same time
there is interest in exploring pragmatic ways to improve the situation
of math and science on the web, especially in the context of recent
developments such as the Houdini Task Force and modularization efforts
in ARIA (such as the DPUB-ARIA module).

In summary, we believe the improved browser technology landscape as well
as our own progress over the past two years provide an opportunity to
re-think MathJax's core component on a fundamental level. We believe
these decisions have to be connected to MathJax's mission. We need to
re-evaluate the current goal of native MathML support in browsers and
consider shifting the focus towards other pragmatic goals that help math
and science become first class citizens on the web.

\section{Goals}\label{goals}

For a redesign of MathJax's core component it is crucial to re-evaluate
our long-term direction. This is where we find ourselves at a
cross-roads.

A critical problem today is that MathJax generates output that currently
can only replace MathML in the page, not augment it. This means we
cannot leverage partial browser implementations (e.g., implementation of
roots or tables) and we cannot provide a positive feedback loop for
browser implementations. In other words, MathJax's current design cannot
function like a polyfill/prolyfill should (in the sense of the
\href{https://extensiblewebmanifesto.org/}{Extensible Web Manifesto}):
providing an implementation in JavaScript that can gradually be replaced
with native browser implementations.

\subsection{The core work: revisiting MathJax's core
component}\label{the-core-work-revisiting-mathjaxs-core-component}

We need to redesign MathJax's modular structure and the core APIs
derived from it to improve performance, re-use and long-term maintenance
of MathJax.

The need to revise MathJax's modular structure relates to current and
future best practices in web development. While MathJax's modular
extension system has allowed MathJax to gradually upgrade older
components as well as develop new, modern components, the modularity has
been ``internal'' only; MathJax components cannot be used outside of
MathJax and developers have to understand MathJax's custom module system
and programming model to modify or build components.




\begin{table}
\tiny
\begin{center}
  \begin{tabularx}{\textwidth}{p{2cm} p{5cm} p{5cm}}
  NEED & MOTIVATION & TOOLS\\
    \hline      
    Scaffold & 
    Several tools. Several ways. Several Practices. Need to organize, and give some good foundation - best practices, good design. & 
    yeoman, Seed Projects, Html5Boilerplate, bootstraps (e.g. Twitter Bootstrap) \\
  
    Build / Automation & 
    Lots of tasks to execute. Compile. Test. Minify. Concat. Etc. & 
    grunt gulp, broccoli, component, …ake's (e.g. Make, Rake, etc.) \\
  
    Automation Utilities & 
    Tasks that can be put in build the pipeline. & 
    minify, uglify, lint, jshint, watch \\
  
    Dependency Management & 
    Applications are getting complex. They rely on several other libraries and frameworks. & 
    bower, component, NPM \\
  
    Dynamic Loading & 
    Big projects are split among several pieces of js for the sake of modularization. No all of them should be loaded at the same time. & 
    require, curl, amd.js, async.js \\
  
    Javascript Preprocessor & 
    The way you organize code in development time is different the way you publish your code. Need to do some processing in your javascript files before using them. & 
    browserift, webpack \\
  
    Application & 
    Applications on web are getting complex, need for frameworks that support app development. & 
    angular, backbone, ember, knockout \\
  
    Application Utilities & 
    Several application features that can be necessary (e.g. routing) & 
    page, director, crossroads2 \\
  
    Test Runner & 
    Execute and visualize test results & 
    karma, saucelabs \\
  
    Test Framework & 
    Write tests & 
    jasmine, mocha, qunit \\
  
    Test End to End & 
    Write tests for the whole application flow & 
    protractor, casperjs, nightwatch.js, watir webdriver \\
  
    Test Support & 
    Support tests and helpers & 
    phantomjs, zombie.js, sinon, chai \\
  
    Dom Utilities & 
    DOM selection and maniputation, some auxiliary functions, need for utilities that make work simple (and cross-browser) & 
    jquery, zepto, polymer, prototype \\
  
    JS Utilities & 
    Clean code, functional programming style, reactive programming features, helpers and utilities & 
    lodash, underscore, promise, fn.js, q.js, bacons.js, sugar.js, chance.js, moment.js, micro.js \\
  
    CI & 
    Continuous integration, continuous delivery, continuous deployment & 
    Any! (e.g. travis ci, jenkins, concrete, semaphore, go, snap) \\
  
    Language & 
    Have a syntactic sugar element, or even completelly different syntax (that in the end turn into javascript to run in the browser) & 
    coffeescript, clojurescript, typescript 3 \\
  
    CSS Preprocessors & 
     & 
    sass, less \\
  
    Preprocessors Libs & 
     & 
    compass, bourbon \\
  
    CSS Helpers & 
     & 
    susy, zenGrids, neat, normalize, modernizr, flexbox \\
  
    CSS Frameworks & 
     & 
    boostrap, foundation, skeleton

  \end{tabularx}
  \caption[]{From Slide 1-3, \href{http://www.slideshare.net/bymarkone/the-javascript-toolkit-20}{``The
JavaScript Toolkit 2.0''}.}
  \label{table:1}
\end{center}
\end{table}


We need to revise this to better address the use cases of modern web
developers. To get a rough idea of the complex workflow and tool chain
into which MathJax to fit, consider \autoref{table:1}.


We need to ensure that MathJax fits better into this diverse ecosystem.
At the heart of this problem lie modules.

Originally and for lack of available technology, MathJax had to create
its own module system, including dynamic module loading mechanisms and
webfont detection. These elaborate internals made MathJax more akin to a
full-fledged web framework rather than a ``regular'' JavaScript library.
In other words, developers had to adapt to MathJax's framework rather
than having a utility library that they can integrate into their
framework of choice. This made the integration work of developers more
complex than one would expect today.

In the past few years, several module systems (e.g., CommonJS, AMD, UMD)
have given rise to native modules in JavaScript itself (starting with
ECMAscript 2015). The progress we have made on MathJax in 2014/2015
allows us to redesign our modular structure so as to simplify it and
turn more components into independent, reusable components using modern
best practices. Such a redesign must focus on web developers who need to
fine tune the components they ship and integrate them into modern tool
chains.

We can easily continue to supply compiled packages allowing the average
users to use MathJax as they do today -- inserting one line of
JavaScript into their header and ``forgetting'' about it.

But by making our components easier to reuse outside of MathJax, we
fulfill our mission better, enable developers to use MathJax more
flexibly, and lower the threshold for outside contributors.

Alongside our module structure, we want to simplify our APIs so as to
make it easier for developers to integrate MathJax into their
applications. Currently, MathJax acts more like a framework, with a
complex callback, signaling, and queuing system and several internal
optimization methods. The problems MathJax solves this way are now much
more common place in web development and most developers will have their
own tools or frameworks to handle these problems (e.g., reducing paints,
reflows and DOM manipulation). This allows us to simplify those APIs by
moving the burden from MathJax towards developers since they already
carry that burden anyway. At the same time, we avoid MathJax's own
methods from interfering with the developers' choices.

To some degree, this kind of change will make MathJax less flexible in
certain use cases (e.g., dynamic adaption to client configuration,
output switching, font switching etc.) and more importantly this will be
heavily influenced by the overall direction discussed below.

To compensate, we can build a developer kit of sample integrations that
implement this old functionality. Ideally, a wider community of
developers will share sample integrations into the most common tools and
frameworks as we've seen with similar situations in the past (such as
mobile apps or CMS integration).

The key difference for MathJax will be a shift towards making additional
flexibility an opt-in for developers and users, removing negative
effects on our core requirements. These technical priorities must be
governed by our overall design direction. We evaluated several
approaches and identified two main candidates.

\subsection{\texorpdfstring{Path 1 ``Seamless
polyfilling''}{Path 1 Seamless polyfilling}}\label{path-1-seamless-polyfilling}

As mentioned, we need to decide if our mission should remain focused on
enabling browser development. If this remains the focus, then we should
re-design MathJax to provide a positive feedback loop for MathML
implementations. The strategy for this approach would be to become
``invisible'' to both users and developers.

This approach would be marked by using newly established webstandards.
For example, we would need to

\begin{itemize}
\itemsep1pt\parskip0pt\parsep0pt
\item
  use custom elements and shadowDOM to make MathJax output appear as its
  underlying MathML,
\item
  design our components to detect partial MathML features and use them
  when available (e.g., roots, menclose, mstyle),
\item
  use mutation observers on the MathML lightDOM as new core API,
\item
  so that developers do not interact with MathJax, they just manipulate
  MathML in the DOM,
\item
  make MathJax fast enough so that developers are comfortable to
  ``forget'' about MathJax rendering, i.e., they inject MathML and
  rendering is seamless and non-interfering.
\end{itemize}

The advantage of this approach lies in the fact that web components has
gained enthusiastic support and the problems MathJax would be facing are
to some degree shared with more developers. Additionally, once even
partial MathML implementations are available, MathJax can improve
automatically.

The risks of this approach lie in the complexity of the task. This kind
of functionality is far more complex than what MathJax does today and it
eliminates many tricks we use in MathJax to ensure high quality and fast
rendering speed. This approach would also have to rely on
computationally heavy browser technology (such as mutation observers,
getComputedStyle etc). It is unclear at this point whether web
components can help with accessibility issues related to polyfilling
MathML since custom elements cannot redefine HTML5 elements and
assistive technology would still be exposed to the shadow tree (cf.~the
latest
\href{http://www.w3.org/TR/shadow-dom/\#h-assistive-technology}{Shadow
DOM working draft}).

The main question is: can we speed up MathJax sufficiently? And would
this modus operandi actually be aligned with web development practices?
Finally, this approach would also eliminate any solutions for
environments without JavaScript as custom elements can only be created
using JavaScript.

\subsection{\texorpdfstring{Path 2 ``HTML5
rendering''}{Path 2 HTML5 rendering}}\label{path-2-html5-rendering}

If leveraging MathML browser implementations is not the goal, then we
should design MathJax to provide a fully-equivalent ``interpretation''
of MathML in HTML.

For example we would

\begin{itemize}
\itemsep1pt\parskip0pt\parsep0pt
\item
  design the core component to focus on working independently of the
  client context,
\item
  in particular, make server-side processing a first-class use case
  alongside client-side processing,
\item
  enrich all output to be at least as powerful as native MathML
  rendering,
\item
  leverage web standards that enable HTML and SVG output to embed
  sufficient semantic information and work on improving such standards,
\item
  focus on DOM-independent processing so as to leverage other web
  technologies (webworkers, serviceworkers) and development techniques
  (e.g., virtualDOM),
\item
  provide lightweight client-side tools to enhance the pre-generated
  output.
\end{itemize}

The advantage of this approach is that MathJax can focus on making its
rendering ``just another piece'' within an HTML page or SVG document,
leveraging regular HTML5 structures, and integrating well with other
content. It would also allow MathJax to enable more functionality
outside the client-side browser, allowing for pre-processing, in
particular improve the situation of MathML in non-JS environments such
as ebooks. Pre-processing would also resolve most performance problems
since no JavaScript processing would be necessary on the client-side
browser (but of course would remain equally possible, e.g., for
interactive content).

The greatest risk of this approach would be the potential damage for the
prospects of MathML on the web, turning the focus to MathML as a data
format for HTML and SVG. It would also somewhat complicate the
interaction with accessibility tools which, despite lack of (visual)
rendering in browsers, have begun to focus on handling mathematics only
as MathML in the DOM; as noted earlier, this problem exists in either
approach. We have reached out to the assistive technology community as
well as the Protocols and Formats Working Group to discuss the potential
of improving related web standards and there seems to be interest from
both sides.

However, if one assumes that browser implementations are unattainable,
then the approach of interpreting MathML in HTML5 can benefit standards
development significantly as it would help identify which features of
MathML might be obsolete in an HTML5 setting and which forms of
mathematics are not yet possible in MathML but possible in a HTML5
setting, e.g., diagramatic content. By providing an easily extensible
model, it could allow for a more dynamic development of mathematics on
the web.

Finally, this approach could identify a minimal set of additions to HTML
and CSS that might simplify such an interpretation of MathML. In
particular, it could inform initiatives such as the recently formed
\href{https://wiki.css-houdini.org/}{Houdini Project} on which
elementary rendering APIs are necessary for mathematical and scientific
notation. Ultimately, this approach could work towards ``merging''
MathML into HTML rather than work towards MathML as a separate set of
standards.

\subsection{Moving Forward}\label{moving-forward}

After careful consideration of all alternatives and many fruitful
discussion with the MathJax Steering Committee, the MathJax Sponsors,
and many experts in the community, we decided to follow the second path.

It seems difficult to argue that browser vendors will actively implement
MathML natively in the next 5 years, no matter how much MathJax might do
to encourage such implementations. The longer implementations are
delayed the less likely it becomes that native MathML will be realized
at all, as HTML, CSS and other standards of the Open Web Platform
continue to evolve without MathML taking part in such change.

The complexities of the first path pose a serious challenge as they will
most likely affect MathJax performance negatively. This could only be
compensated with partial native implementations to eventually take over
the difficult layout tasks. However, without such implementations,
MathJax stands to worsen as a product.

The second path moves our focus towards implemented HTML5 standards.
With the new CommonHTML output, developing an ``interpretation'' of
MathML via HTML/CSS and SVG that is as powerful as native MathML is an
attainable goal. Working towards HTML/CSS improvements that simplify
this ``interpretation'' seems much more likely to succeed.

This path also makes server-side processing a first class use case for
MathJax. The ability to pre-generate rendering resolves performance
issues as JavaScript is no longer necessary on the client -- yet MathJax
will still work equally well on the client.

Finally, the second path would still allow for client-side enhancements
that realize the first approach (i.e., using web components to turn the
output into faux MathML) while the reverse does not seem possible.
Overall, native MathML support, while ideal, seems unrealistic, whereas
an ``interpretation'' of MathML is a pragmatic approach with equal
potential.

Another important consideration is future expandability. It is clear
that other scientific, technical or artistic content will need a more
natural representation on the web. Indeed, there already exist a number
of specialist languages for such content (e.g., CML, MusicXML,
PhyloXML) that might push for inclusion in web content in the future.
Similarly, math on the web pushes beyond MathML on various ends, from
geometric representation to computer algebra systems to new forms of
visual expression of mathematical thought.

The MathML experience teaches us that not even inclusion into W3C
standards leads to implementation of specialist languages in browsers.
Given that HTML5 is the basis for ebook standards such as ePub3 and
mobi/kindle, it also means that every eBook reader needs to implement
the full spec. But if browser vendors already do not implement the full
standard, it is even less likely that developers of eBook readers will.
Consequently, there is always a necessity for polyfill solutions like
MathJax and there is an argument to be made that this will even increase
in the future, with other markup languages reaching maturity. And
equally consequently, any progress towards realizing MathML and
mathematical knowledge in general in HTML and SVG could provide a path
for other knowledge domains.

We see a future market in MathJax to provide rendering solutions for
these scientific languages and consequently need a flexible and future
proof underpinning. That means firstly we want to restrict the rendering
output to those parts of the web standard that are widely implemented in
all display solutions, i.e.~HTML, CSS and SVG. And secondly, we need a
flexible and efficient core system that is based on modern JavaScript
standards and that can be more easily adapted and updated to future
iterations of web applications technology.

\section{Methods}\label{methods}

We want to update our core component and modular infrastructure using
current and future best practices of JavaScript development. The
redesign should therefore also bring about a change in MathJax's
development process.

In this context, it is important to realize that we are considering
changes that will hold back development for a signficant amount of time,
most likely a full year. However, by modularizing MathJax more
aggressively and by following a more dynamic development process, we can
release early and often, focusing on smaller and independent releases of
our components that developers can integrate rapidly. In particular, we
will not strive to deliver an equivalent product from the start but
quickly provide incremental releases of individual components.

This change also includes building more on other polyfills and
JavaScript compilers/transpilers. This will allow us to develop MathJax
for the browser market of 2016 and 2017 rather than the lowest common
denominator of 2015.

Because the web development ecosystem has become very diverse, we are
painfully aware that we cannot hope to keep track of all current and
emerging technologies. That is why we will form a Technical Committee
consisting of developers from our sponsors, our contributors, and other
specialists. Building a strong connection with this dedicated group of
developers will guide our design decisions towards the broadest positive
impact.

Finally, we will continue to maintain MathJax 2.x in terms of bug fixes
and third party contributions.

\section{Resources}\label{resources}

To make this change, we need to expand our team to permanently have two
equal core contributors. For this, we need to ensure that our
development schedule can be aligned, e.g., in the form of dedicated
development sprints. This includes aligning our work with the members of
the technical committee, who we will rely on for providing feedback
during development sprints.

\section{Measuring success}\label{measuring-success}

As we pursue a path towards an HTML5 ``interpretation'', the primary
measure of success will be how well MathJax can provide a solution for
mathematics on the web that is at least equivalent to native MathML
browser implementations, for both developers and end users and in
particular in terms of accessibility. Our internal approach of using
MathML will not change but its role in our rendering is open to change
as we work towards realizing MathML in HTML. We are dedicated to the
Open Web Platform and its standards and our work will continue to be
focused on providing feedback to standards and browser development;
therefore progress on web standards will also be a measure of success.

Rendering speed will naturally be a critical measure, as will be overall
functionality (i.e., MathML coverage and more advanced math and science
use case such as diagrams and responsive rendering).

Finally, a timeline developed by the new team (with input from the
technical committee) will provide further means of tracking progress in
the first quarter of 2016.

\section{Conclusion}\label{conclusion}

We believe we have reached a point in time where we could and should
undertake significant changes to MathJax's core component to adapt to a
changed landscape. The nature of these changes depend on a critical
decision for the overall direction of MathJax's mission -- to no longer
consider native MathML support in browsers an achievable goal of
MathJax. We believe we have identified the relevant technologies to
successfully make this change and ensure that MathJax greatly improves
and serves the community for the next 5 years.

We have sought out our sponsors and other members of the MathJax
community to ensure we have their input and support for this change. As
part of this change, a group of expert developers will support us as a
technical advisory group that will advise us in this effort. Thanks to
the unanimous support of our MathJax sponsors we look forward to
tackling this change towards greatly improving MathJax as a long
lasting, high quality tool for the entire community.

\end{document}
